\documentclass{article}
\usepackage{amsmath, amssymb}
\usepackage{listings}

\lstset{
  basicstyle=\ttfamily,
  showstringspaces=false
}
\newcommand{\code}[1]{\lstinline`#1`}





\begin{document}
The repository defines a compiler from LaTeX-mixed-with-J to LaTeX code.
The script jlatex.ijs defines the verb \code{latex} to convert a J phrase
into a well-formatted LaTeX math mode string.  This is then used in a
compiler, which converts a file (I use .jtex as an extension, although
it doesn't matter) into a .tex document and compiles that with pdflatex.
Sections enclosed with \code{\J()} are replaced with their results
from the \code{latex} function.

The script allows J to function as it usually does, but makes changes to
the way arithmetic operators work. It manipulates both J variables and
LaTeX expressions. Each LaTeX expression is contained in a box, so it can
be treated as an atom and used in arrays without concern for details.
Functions modified to produce LaTeX results will convert their inputs
to LaTeX before operating on them if necessary.

The syntax is precisely J's, except for the special character
\lstinline`\`, which works as follows:
\begin{itemize}
  \item \lstinline`\const`, where \code{const} is a number or string,
    converts that value to a LaTeX expression.
  \item \lstinline`\fun`, where \code{fun} begins with an alphabetic,
    gives a LaTeX function, which will apply as \lstinline`\fun{y}`
    or \lstinline`\fun{x}{y}`.
  \item \lstinline`\_val` gives a LaTeX literal \lstinline`\val`, which is a noun.
  \item \lstinline`\\op` gives a verb which functions as an operator,
    applying as \lstinline`\op y` or \lstinline`x \op y`.
  \item \lstinline`\op`, where \code{b} starts with a special character or
    ends with one of \code{.:}, gives the J function b.
  \item \lstinline`\(expr)` (with parentheses) interprets \code{(expr)}
    as a J expression and does not change it.
\end{itemize}

Also, the new primitive \code{(.} forces parentheses around an expression.

The code redefines arithmetic functions (the noun OPS gives the full list)
to give LaTeX-formatted outputs. However, all other functions will work as
they do in J, provided they are applied to arguments which have not yet
been formatted in LaTeX (i.e. \code{i.4} will work fine, but \code{i.4+5}
will give a domain error). Adverbs, conjunctions, hooks, and forks perform
excellently, including rank.

Names which are not defined at execution are changed into LaTeX literals.
This allows the use of J variables, while making use of literals easier. 
A name or string can always be enclosed in quotes if needed.
One particular case is as the target of assignment---names must be quoted
or J will return a domain error.

Parentheses are added when necessary and usually only when necessary---
the converter is aware of order of operations, associativity, and which
operations need parentheses (i.e. \code{+} does, \code{^} needs them
only on the left, and \code{\%} needs none).

The following are a few examples of J code converted to LaTeX.
\begin{lstlisting}
(2*a)%~(-b)\\pm\sqrt(b^2)-4*a*c
\end{lstlisting}
\[\frac{-b \pm \sqrt{b^2-4 a c}}{2 a}\]
\begin{lstlisting}
+/ *: a,b,c
\end{lstlisting}
\[a^2+b^2+c^2\]
\begin{lstlisting}
\_pi * 2*(rh+r^2)
\end{lstlisting}
\[\pi\cdot 2 (rh+r^2)\]
\begin{lstlisting}
(+1&\cfrac)/ (}:,+&\_ddots@{:) a_"0 i.5
\end{lstlisting}
\[a_0+\cfrac{1}{a_1+\cfrac{1}{a_2+\cfrac{1}{a_3+\cfrac{1}{a_4+\ddots}}}}\]
\end{document}
