\documentclass{article}
\usepackage{amsmath, amssymb}
\usepackage{../tutorial}



\begin{document}
The \lstinline`math` addon provides ways of writing commonly used mathemetical
operations and symbols. It is accompanied by scripts for more specific
uses, currently \lstinline`math/func.ijs` and \lstinline`math/sets.ijs`.

The \lstinline`array` addon is loaded by \lstinline`math`, and is used freely here. If you
are unfamiliar with the array functions used here, see the array tutorial.

\section{Greek}
The \lstinline`greek` addon defines nouns for all the greek letters. It is used
here for convenience.
\begin{lstlisting}
  tau = +:pi
\end{lstlisting}
\[\tau=2 \pi\]
\begin{lstlisting}
  mathlist Gamma, Delta, Theta
\end{lstlisting}
$\Gamma$, $\Delta$, and $\Theta$

\section{Mathematical functions}
The \lstinline`of` verb is the same as math's function application. It can be
applied to a list of arguments as well. \lstinline`of` always parenthesizes the
argument.
\begin{lstlisting}
  f of x
\end{lstlisting}
\[f(x)\]
\begin{lstlisting}
  f of r,theta,phi
\end{lstlisting}
\[f(r,\theta,\phi)\]
\begin{lstlisting}
  (Gamma of 3r2) = -:%:pi
\end{lstlisting}
\[\Gamma\left(\frac{3}{2}\right)=\frac{\sqrt{\pi}}{2}\]

With large arguments functions use \lstinline`\left` and \lstinline`\right` for their
delimiters, which results in poor spacing without further configuration.
To avoid this, add the following lines (also found in \lstinline`../tutorial.sty`)
to the preamble of your document:
\begin{verbatim}
\let\originalleft\left
\let\originalright\right
\renewcommand{\left}{%
    \mathopen{}\mathclose\bgroup\originalleft}
\renewcommand{\right}{\aftergroup\egroup\originalright}
\end{verbatim}

\section{Math operation types}
Three new types of operations are defined by \lstinline`math`:

\begin{center}\begin{tabular}{lccl}
Style & \multicolumn{2}{c}{Example}    & Declaration \\ \hline
Sum   & \lstinline`sum` & $\sum_j a_j$        & \lstinline`DeclareSumOp` \\
Math  & \lstinline`sin` & $\sin \theta$        & \lstinline`DeclareMathOp` \\
Limit & \lstinline`lim` & $\lim_{n \rightarrow \infty}a_n$ & \lstinline`DeclareMathLimOp`
\end{tabular}\end{center}

In each case, nearly all operations built into LaTeX are remapped. If you
need one that is not or have defined your own operation, simply use the
appropriate declaration verb from the right column.

\subsection{Sum-style operations}
Functions like the big-sigma sum and integrals take a bound as a left
argument and the term to sum (integrate, etc.) as the right argument.
The bound has length up to two: the first element is the lower bound and
the second is the upper bound. It may also be omitted entirely.
\begin{lstlisting}
  (6%~*:pi) =~ (\_infty ,~ n=1) sum %*:n
\end{lstlisting}
\[\sum_{n=1}^{\infty} \frac{1}{n^2}=\frac{\pi^2}{6}\]
\begin{lstlisting}
  X = (alpha in A) bigcup U_ alpha
\end{lstlisting}
\[X=\bigcup_{\alpha \in A} U_{\alpha}\]
\begin{lstlisting}
  (int@:(*&dx) = +&C) ^x
\end{lstlisting}
\[\int e^{x} dx=e^{x}+C\]

\subsection{Math operations}
Operations like \lstinline`sin`, \lstinline`exp`, and so on are defined as verbs. For
these functions, the argument is parenthesized only if it is sufficiently
complicated.
\begin{lstlisting}
  list sin"0 (+: , ] , -:) theta
\end{lstlisting}
\[\sin (2 \theta),\sin \theta,\sin \left(\frac{\theta}{2}\right)\]
\begin{lstlisting}
  =`+/ dim"0 V ([ , ker@] , of~) T
\end{lstlisting}
\[\dim V=\dim (\ker T)+\dim (T(V))\]
\begin{lstlisting}
  (gcd = \+./) 42 98 126
\end{lstlisting}
\[\gcd (42,98,126)=14\]

Note that \lstinline`log` is not defined---use \lstinline`^.` instead!

\subsection{Limit-style operations}
Limit operations are monadic or take a single left argument, which is
placed as a subscript.
\begin{lstlisting}
  min ((-~^) st in&RR) x
\end{lstlisting}
\[\min \left\{\,e^{x}-x\,\middle|\,x \in \mathbb{R}\,\right\}\]
\begin{lstlisting}
  (n to _) lim | a %/@:\_ (,<:)n
\end{lstlisting}
\[\lim_{n \rightarrow \infty}\left|\frac{a_n}{a_{n-1}}\right|\]

\section{Miscellaneous}
The operations \lstinline`oplus` and \lstinline`otimes` are declared in \lstinline`math`:
\begin{lstlisting}
  b ((a otimes [.@oplus) = oplus&[.&(a&otimes)) c
\end{lstlisting}
\[a \otimes (b \oplus c)=(a \otimes b) \oplus (a \otimes c)\]

\section{Sets}
The \lstinline`math/sets` addon provides some standard set nouns as well as verbs
for common ways to write sets.

The following sets are predefined:
\begin{center}
  \begin{tabular}{lc}
    \lstinline`NN` & $\mathbb{N}$ \\
    \lstinline`ZZ` & $\mathbb{Z}$ \\
    \lstinline`QQ` & $\mathbb{Q}$ \\
    \lstinline`RR` & $\mathbb{R}$ \\
    \lstinline`CC` & $\mathbb{C}$ \\
    \lstinline`II` & $[0,1]$
  \end{tabular}
\end{center}

The \lstinline`DeclareSet` verb is the declaration for a set, which is written in
blackboard bold.
\begin{lstlisting}
  HH DeclareSet H
  HH in~ list \1,i,j,k
\end{lstlisting}
\[1,i,j,k \in \mathbb{H}\]

To place a list of elements in a set, use the \lstinline`set` verb. To invoke
``set-builder notation,'' use \lstinline`st`. The left argument to \lstinline`st` is the
variable and the right argument is the predicate.
\begin{lstlisting}
  set \(/:~ , */&:>/ (^ i.@>:)&.>/ __ q: 12)
\end{lstlisting}
\[\{1,2,3,4,6,12\}\]
\begin{lstlisting}
  (x in RR) st (*:x) < 2
\end{lstlisting}
\[\left\{\,x \in \mathbb{R}\,\middle|\,x^2<2\,\right\}\]

The verbs \lstinline`subset`, \lstinline`supset`, \lstinline`in`, and \lstinline`setminus` are defined as
math operations.
\begin{lstlisting}
  iff DeclareOp 'Longleftrightarrow'
  A (subset iff supset~ iff \_varnothing =~ setminus) B
\end{lstlisting}
\[A \subset B \Longleftrightarrow B \supset A \Longleftrightarrow A \setminus B=\varnothing\]

The infix operations \lstinline`and`, \lstinline`quot`, \lstinline`by`, and \lstinline`comma` are defined.
\begin{lstlisting}
  (x and y) in S
\end{lstlisting}
\[x,y \in S\]
\begin{lstlisting}
  (quot p&*) ZZ
\end{lstlisting}
\[\mathbb{Z}/p \mathbb{Z}\]
\begin{lstlisting}
  int (^x) by dx  NB. space with \,
\end{lstlisting}
\[\int e^{x}\,dx\]
\begin{lstlisting}
  (a=3) comma (b=5)  NB. space with \quad
\end{lstlisting}
\[a=3,\quad b=5\]

\section{Functions}
The following functions are defined by \lstinline`math/func`.

The operations \lstinline`from` (as \lstinline`colon`), \lstinline`to` (as \lstinline`rightarrow`), and
\lstinline`mapsto` are declared:
\begin{lstlisting}
  (f from RR to RR)  comma  x mapsto *:x
\end{lstlisting}
\[f \colon \mathbb{R} \rightarrow \mathbb{R},\quad x \mapsto x^2\]

Function composition is \lstinline`o`.
\begin{lstlisting}
  ((f o g)&of = f of g&of) x
\end{lstlisting}
\[f \circ g(x)=f(g(x))\]

The inverse is \lstinline`inv`, and you can apply a function's inverse with
\lstinline`if` (that is, \lstinline`of` with the first character changed).
The function \lstinline`res` gives the restriction of a function to a domain,
using a vertical bar.
\begin{lstlisting}
  a (* * *&inv) b
\end{lstlisting}
\[a b a^{-1} b^{-1}\]
\begin{lstlisting}
  ([.f res A) if x
\end{lstlisting}
\[\left(f|_A\right)^{-1}(x)\]

The verbs \lstinline`star` and \lstinline`prime` do the obvious:
\begin{lstlisting}
  (star H_1) of S^2
\end{lstlisting}
\[H_1^*\left(S^2\right)\]
\begin{lstlisting}
  (prime^:2 f) of x
\end{lstlisting}
\[f''(x)\]

The conjunction \lstinline`u commutes v` gives \lstinline`u@:v = v&:u`.
\begin{lstlisting}
  *: commutes % x
\end{lstlisting}
\[\left(\frac{1}{x}\right)^2=\frac{1}{x^2}\]
\begin{lstlisting}
  z + commutes \\cdot w
\end{lstlisting}
\[\overline{z \cdot w}=\overline{z} \cdot \overline{w}\]
\end{document}
